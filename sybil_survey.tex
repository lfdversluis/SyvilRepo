% THIS IS SIGPROC-SP.TEX - VERSION 3.1
% WORKS WITH V3.2SP OF ACM_PROC_ARTICLE-SP.CLS
% APRIL 2009
%
% It is an example file showing how to use the 'acm_proc_article-sp.cls' V3.2SP
% LaTeX2e document class file for Conference Proceedings submissions.
% ----------------------------------------------------------------------------------------------------------------
% This .tex file (and associated .cls V3.2SP) *DOES NOT* produce:
%       1) The Permission Statement
%       2) The Conference (location) Info information
%       3) The Copyright Line with ACM data
%       4) Page numbering
% ---------------------------------------------------------------------------------------------------------------

\documentclass{acm_proc_article-sp}
\usepackage{enumitem}% http://ctan.org/pkg/enumitem
\usepackage{footmisc}
\usepackage{makecell}
\usepackage{longtable}
\makeatletter
\newcommand\footnoteref[1]{\protected@xdef\@thefnmark{\ref{#1}}\@footnotemark}
\makeatother

\begin{document}

\title{A Survey of Real Sybil Attacks}
%\subtitle{[Extended Abstract]
%\titlenote{A full version of this paper is available as
%\textit{Author's Guide to Preparing ACM SIG Proceedings Using
%\LaTeX$2_\epsilon$\ and BibTeX} at
%\texttt{www.acm.org/eaddress.htm}}}
%
% You need the command \numberofauthors to handle the 'placement
% and alignment' of the authors beneath the title.
%
% For aesthetic reasons, we recommend 'three authors at a time'
% i.e. three 'name/affiliation blocks' be placed beneath the title.
%
% NOTE: You are NOT restricted in how many 'rows' of
% "name/affiliations" may appear. We just ask that you restrict
% the number of 'columns' to three.
%
% Because of the available 'opening page real-estate'
% we ask you to refrain from putting more than six authors
% (two rows with three columns) beneath the article title.
% More than six makes the first-page appear very cluttered indeed.
%
% Use the \alignauthor commands to handle the names
% and affiliations for an 'aesthetic maximum' of six authors.
% Add names, affiliations, addresses for
% the seventh etc. author(s) as the argument for the
% \additionalauthors command.
% These 'additional authors' will be output/set for you
% without further effort on your part as the last section in
% the body of your article BEFORE References or any Appendices.

\numberofauthors{1} %  in this sample file, there are a *total*
% of EIGHT authors. SIX appear on the 'first-page' (for formatting
% reasons) and the remaining two appear in the \additionalauthors section.
%
\author{
\alignauthor
Laurens Versluis\\
       \affaddr{Delft University of Technology}\\
       \affaddr{Delft, The Netherlands}\\
       \email{L.F.D.Versluis@student.tudelft.nl}
}

\maketitle

\begin{abstract}
	

\end{abstract}

% A category with the (minimum) three required fields
%\category{H.4}{Information Systems Applications}{Miscellaneous}
%A category including the fourth, optional field follows...
%\category{D.2.8}{Software Engineering}{Metrics}[complexity measures, performance measures]

%\terms{Theory}

%\keywords{Sybils, Sybil Attack, }

\section{Introduction}
	\label{sct:introduction}
	The sbyil attack was first described by Douceur \cite{douceur2002sybil}. Nowadays, it is a well-known attack on both centralized and decentralized systems and an active research area.
	In the sybil attack, malicious users create an unbounded number of \emph{sybil} identities.
	Using these sybils, malicious users can perform several attacks. 
	Hoffman et al. (2009) \nocite{hoffman2009survey} identified five classes of attacks: 
	\begin{enumerate}
		\item Self-promoting: Attackers boost their own reputation or increase their gain/profit.
		\item Whitewashing: Attackers avoid consequences of abusing the system and repair their own reputation to continue their attacks.
		\item Slandering: Attackers manipulate the reputation of other users inside the system by e.g. false reports.
		\item Orchestrated: A combination of the three attacks mentioned above.
		\item Denial of Service: Attackers prevent the calculation and dissemination of reputation values. 
	\end{enumerate}
	Often these sybils are indistinguishable from real users and therefore a threat to systems relying on user input.
	Decentralized systems are particularly vulnerable.
	Without a central authority to certify users, decentralized systems are vulnerable to a variety of attacks, including the sybil attack \cite{jetter2010quantitative}.
	Douceur argues that a central authority which certifies all identies may tbe the only effective solution, however even when a centralized authority is present, it still may not be feasible to certify real users and can even compromise the anonymity of peers \cite{margolin2005quantifying, dewan2005securing}.
	
	Sybil attacks occur in a variety of systems and networks such as overlay networks \cite{singh2004defending}, social networks \cite{cao2014understanding, chiluka2012leveraging, mohaisen2011keep}, content rating systems \cite{kakhki11mitigating, tran2009sybil} and vehicular ad hoc networks \cite{park2009defense}.
	Since the problem is widespread and many solutions have been proposed, there are already surveys which compare different solutions/surveys \cite{koll2014state, mohaisen2013sybil, viswanath2011analysis}.\\
	The focus of this survey will not be yet another survey on the current state of the art, but will focus on real-world attacks using Sybil, eclipse and sinkholing techniques. 
	We perceive these to be belonging to the same broad class of attacks. 
	
	%The goal is to provide a list of scientific articles which are based on a publicly available real-world datasets.
	This survey has two goals.
	The first goal is to provide a list of scientific articles and describe the datasets used in their evaluations.
	Sybils can be assigned a taxonomy as they can be compromised nodes, fake nodes \cite{newsome2004sybil} or whitewashed nodes, however we do not make this distinction inside datasets.  
	The second goals is to present the largest structured collection of various sybil network datasets which are collected if data is publicly available or if the authors are willing to share their data. 
	Additional datasets are added as well which were either created by means of manual annotation or by other parties.\\
	The list of datasets will, for instance, cover fake profiles on social networking sites (Facebook), communication systems (Twitter), search engine link farms, auction sites, review sites, sock puppets on news sites, and various other Internet-deployed systems. 
	A key challenge is the diversity and formatting of these datasets. 
	The aim is to design a unifying format to enable scientists to easily use all available datasets for their latest research findings with minimal effort.
	This survey will provide a structured listing with key aspects of each dataset, including, description, origin, size, creation date, and copyright license.

	% TODO: structure?
\section{Datasets}
	
	In this section, the current state of the art on Sybil attacks and their datasets is reviewed.
	We list well-known papers on the sybil attack and list several aspects including the year, size, amount of sybils, real or artificial data and availability of the dataset.
	
	\begin{table*}
		\centering
		\begin{tabular}{|c|l|l|l|l|l|}
			\hline
			Year & Mechanism & \# Nodes & \# Sybils & Real-world data & Dataset availability \\ \hline
			
			2004 & Overlay defense$^*$ \cite{singh2004defending} & 5050 & 1010 & No & 
			\begin{minipage}{1.2in}
			\vskip 1pt
			\begin{itemize}[noitemsep,topsep=0pt,leftmargin=*]
				\item No link in paper
				\item Public availability unknown
				\item Author response pending
			\end{itemize}
			\vskip 1pt
			\end{minipage}  \\ \hline
		
			2005 & Defending sensors$^*$ \cite{zhang2005defending} & No simulation & No simulation & N/A & N/A \\ \hline
			
			2006 & Self-registration$^*$ \cite{dinger2006defending} & $\pm 500$ & $\pm 20$ & No & 
			\begin{minipage}{1.2in}
				\vskip 1pt
				\begin{itemize}[noitemsep,topsep=0pt,leftmargin=*]
					\item No link in paper
					\item Public availability unknown
					\item Author response pending
				\end{itemize}
				\vskip 1pt
			\end{minipage}  \\ \hline
			
			2006 & SybilGuard \cite{yu2006sybilguard} & 
			\begin{minipage}{0.9in}
				\vskip 1pt
				\begin{enumerate}[noitemsep,topsep=0pt,leftmargin=*]
					\item 1.000.000
					\item 10.000
					\item 100
				\end{enumerate}
				\vskip 1pt
			\end{minipage} & $\pm 100$ & No & \begin{minipage}{1.2in}
			\vskip 1pt
			\begin{itemize}[noitemsep,topsep=0pt,leftmargin=*]
				\item No link in paper
				\item Public availability unknown
				\item Author response pending
			\end{itemize}
			\vskip 1pt
			\end{minipage}  \\ \hline
			
			2006 & Computational Puzzles \cite{borisov2006computational} & No simulation & No simulation & N/A & N/A \\ \hline
			
			2008 & Sybillimit \cite{yu2008sybillimit} &
			\begin{minipage}{0.9in}
				\vskip 1pt
				\begin{enumerate}[noitemsep,topsep=0pt,leftmargin=*]
					\item 932.512
					\item 900.822
					\item 106.002
					\item 1.000.000
				\end{enumerate}
				\vskip 1pt
			\end{minipage} 
			& TBD &
			\begin{minipage}{0.9in}
				\vskip 1pt
				\begin{enumerate}[noitemsep,topsep=0pt,leftmargin=*]
					\item Yes
					\item Yes
					\item Yes
					\item No
				\end{enumerate}
				\vskip 1pt
			\end{minipage} 
			&
			\begin{minipage}{1.2in}
				\vskip 1pt
				\begin{itemize}[noitemsep,topsep=0pt,leftmargin=*]
					\item No link in paper
					\item Public availability unknown
					\item Author response pending
				\end{itemize}
				\vskip 1pt
			\end{minipage}  \\ \hline
			
			2008 & Cluster Analysis$^*$ \cite{yang2008detecting} &
			\begin{minipage}{0.9in}
				\vskip 1pt
				\begin{enumerate}[noitemsep,topsep=0pt,leftmargin=*]
					\item 101
					\item 94
				\end{enumerate}
				\vskip 1pt
			\end{minipage} & \makecell[l]{ All possible pairs:\\ 1. 5.050 \\ 2. 4.371} & \makecell[l]{Yes \\ (Since it concerns real \\ devices in this paper,\\we perceive it as real\\ data)} &
			\begin{minipage}{1.2in}
				\vskip 1pt
				\begin{itemize}[noitemsep,topsep=0pt,leftmargin=*]
					\item No link in paper
					\item Public availability unknown
					\item Author response pending
				\end{itemize}
				\vskip 1pt
			\end{minipage}  \\ \hline
			
			2009 & SybilInfer \cite{danezis2009sybilinfer} &
			\begin{minipage}{0.9in}
				\vskip 1pt
				\begin{enumerate}[noitemsep,topsep=0pt,leftmargin=*]
					\item 1.000
					\item $\pm$33.000
				\end{enumerate}
				\vskip 1pt
			\end{minipage}
			& 
		 	\begin{minipage}{0.9in}
		 		\vskip 1pt
		 		\begin{enumerate}[noitemsep,topsep=0pt,leftmargin=*]
		 			\item 100
		 			\item $\pm$ 2.000
		 		\end{enumerate}
		 		\vskip 1pt
		 	\end{minipage}
		 	 & 
	 	 	\begin{minipage}{0.9in}
	 	 		\vskip 1pt
	 	 		\begin{enumerate}[noitemsep,topsep=0pt,leftmargin=*]
	 	 			\item No
	 	 			\item Yes
	 	 		\end{enumerate}
	 	 		\vskip 1pt
	 	 	\end{minipage}
			&
			\begin{minipage}{1.2in}
			 	\vskip 1pt
			 	\begin{itemize}[noitemsep,topsep=0pt,leftmargin=*]
			 		\item No link in paper
			 		\item Public availability unknown
			 		\item Author response pending
			 	\end{itemize}
			 	\vskip 1pt
			\end{minipage} \\ \hline
			 
			2009 & Timestamp series \cite{park2009defense} & No simulation & No simulation & N/A & N/A \\ \hline
			
			2009 & SyMon \cite{jyothi2009symon} & 50.000  & \makecell[l]{ 2.500 to 25.000 \\ in steps of 2.500}
			& No & 
			\begin{minipage}{1.2in}
			\vskip 1pt
			\begin{itemize}[noitemsep,topsep=0pt,leftmargin=*]
				\item No link in paper
				\item Public availability unknown
				\item Author response pending
			\end{itemize}
			\vskip 1pt
			\end{minipage}  \\ \hline
			
			2009 & Dsybil \cite{yu2009dsybil} & 
			\begin{minipage}{0.9in}
				\vskip 1pt
				\begin{enumerate}[noitemsep,topsep=0pt,leftmargin=*]
					\item 496.622
					\item 2.339
					\item 480.189
					\item 6.040
					\item 105.283
				\end{enumerate}
				\vskip 1pt
			\end{minipage} 
			& Unknown & Yes & 
			\begin{minipage}{1.2in}
				\vskip 1pt
				\begin{itemize}[noitemsep,topsep=0pt,leftmargin=*]
					\item No link in paper
					\item Public availability unknown
					\item Author response pending
				\end{itemize}
				\vskip 1pt
			\end{minipage} \\ \hline
			
			2009 & SumUp \cite{tran2009sybil} & 3.002.907 & \makecell[l]{No ground truth\\ Estimation: 12\% \\(360.349)} & Yes &  
			\begin{minipage}{1.2in}
				\vskip 1pt
				\begin{itemize}[noitemsep,topsep=0pt,leftmargin=*]
					\item No link in paper
					\item Public availability unknown
					\item Author response pending
				\end{itemize}
				\vskip 1pt
			\end{minipage} \\ \hline
			
			2011 & GateKeeper \cite{tran2011optimal} & 
			\begin{minipage}{0.9in}
				\vskip 1pt
				\begin{enumerate}[noitemsep,topsep=0pt,leftmargin=*]
					\item Varying (Synthetic)
					\item 446.181
					\item 539.242
				\end{enumerate}
				\vskip 1pt
			\end{minipage}
			& 
			\begin{minipage}{0.75in}
				\vskip 1pt
				\begin{enumerate}[noitemsep,topsep=0pt,leftmargin=*]
					\item Varying
					\item 43.725 sybils admitted
					\item 76.572 sybils admitted
				\end{enumerate}
				\vskip 1pt
			\end{minipage}
			& 
			\begin{minipage}{0.9in}
				\vskip 1pt
				\begin{enumerate}[noitemsep,topsep=0pt,leftmargin=*]
					\item No
					\item Yes
					\item Yes
				\end{enumerate}
				\vskip 1pt
			\end{minipage}
			&
			\begin{minipage}{1.2in}
				\vskip 1pt
				\begin{itemize}[noitemsep,topsep=0pt,leftmargin=*]
					\item No link in paper
					\item Public availability unknown
					\item Author response pending
				\end{itemize}
				\vskip 1pt
			\end{minipage} \\ \hline
			
			2011 & Mitigating$^*$ \cite{kakhki11mitigating} & 
			\makecell[l]{$> 65.000$\\
					(Sybil network \\attached, no \\information on \\size)}
			& Not mentioned & 
			Yes, real sybils unkown 
			& \begin{minipage}{1.2in}
				\vskip 1pt
				\begin{itemize}[noitemsep,topsep=0pt,leftmargin=*]
					\item No link in paper
					\item Public availability unknown
					\item Author response pending
				\end{itemize}
				\vskip 1pt
			\end{minipage} \\ \hline
			
			2011 & Leveraging$^*$ \cite{chiluka2012leveraging} & 542.133 & 16.264 (3\%) & Yes & 
			\begin{minipage}{1.2in}
				\vskip 1pt
				\begin{itemize}[noitemsep,topsep=0pt,leftmargin=*]
					\item No link in paper
					\item Public availability unknown
					\item Author response pending
				\end{itemize}
				\vskip 1pt
			\end{minipage} \\ \hline
			
			2011 & Incorperating trust$^*$ \cite{mohaisen2011keep} & & & & \\ \hline
			
			2012 & SybilDefender \cite{wei2012sybildefender} &  & & & \\ \hline
			
			2013 & Sok \cite{alvisi2013sok} &  & & & \\ \hline
			
			2013 & SybilShield \cite{shi2013sybilshield} &  & & & \\ \hline
			
			2014 & SybilRank \cite{cao2014understanding} & & & & \\ \hline
			
			
		\end{tabular}
		\caption{Current state of the art reviewed on their datasets. ( * = mechanism was not named by the author(s)).}
		\label{tbl:state-of-the-art-reviewed}
	\end{table*}

%
% The following two commands are all you need in the
% initial runs of your .tex file to
% produce the bibliography for the citations in your paper.
\bibliographystyle{abbrv}
\bibliography{sigproc}  % sigproc.bib is the name of the Bibliography in this case
\balancecolumns
\end{document}
