% THIS IS SIGPROC-SP.TEX - VERSION 3.1
% WORKS WITH V3.2SP OF ACM_PROC_ARTICLE-SP.CLS
% APRIL 2009
%
% It is an example file showing how to use the 'acm_proc_article-sp.cls' V3.2SP
% LaTeX2e document class file for Conference Proceedings submissions.
% ----------------------------------------------------------------------------------------------------------------
% This .tex file (and associated .cls V3.2SP) *DOES NOT* produce:
%       1) The Permission Statement
%       2) The Conference (location) Info information
%       3) The Copyright Line with ACM data
%       4) Page numbering
% ---------------------------------------------------------------------------------------------------------------

\documentclass{acm_proc_article-sp}

\begin{document}

\title{A Survey of Real Sybil Attacks}
%\subtitle{[Extended Abstract]
%\titlenote{A full version of this paper is available as
%\textit{Author's Guide to Preparing ACM SIG Proceedings Using
%\LaTeX$2_\epsilon$\ and BibTeX} at
%\texttt{www.acm.org/eaddress.htm}}}
%
% You need the command \numberofauthors to handle the 'placement
% and alignment' of the authors beneath the title.
%
% For aesthetic reasons, we recommend 'three authors at a time'
% i.e. three 'name/affiliation blocks' be placed beneath the title.
%
% NOTE: You are NOT restricted in how many 'rows' of
% "name/affiliations" may appear. We just ask that you restrict
% the number of 'columns' to three.
%
% Because of the available 'opening page real-estate'
% we ask you to refrain from putting more than six authors
% (two rows with three columns) beneath the article title.
% More than six makes the first-page appear very cluttered indeed.
%
% Use the \alignauthor commands to handle the names
% and affiliations for an 'aesthetic maximum' of six authors.
% Add names, affiliations, addresses for
% the seventh etc. author(s) as the argument for the
% \additionalauthors command.
% These 'additional authors' will be output/set for you
% without further effort on your part as the last section in
% the body of your article BEFORE References or any Appendices.

\numberofauthors{1} %  in this sample file, there are a *total*
% of EIGHT authors. SIX appear on the 'first-page' (for formatting
% reasons) and the remaining two appear in the \additionalauthors section.
%
\author{
\alignauthor
Laurens Versluis\\
       \affaddr{Delft University of Technology}\\
       \affaddr{Delft, The Netherlands}\\
       \email{L.F.D.Versluis@student.tudelft.nl}
}

\maketitle
\begin{abstract}
	

\end{abstract}

% A category with the (minimum) three required fields
%\category{H.4}{Information Systems Applications}{Miscellaneous}
%A category including the fourth, optional field follows...
%\category{D.2.8}{Software Engineering}{Metrics}[complexity measures, performance measures]

%\terms{Theory}

%\keywords{Sybils, Sybil Attack, }

\section{Introduction}

	This survey will focus on real-world attacks using Sybil, eclipse and sinkholing techniques. 
	We perceive these to be belonging to the same broad class of attacks. 
	The goal is to provide a list of scientific articles which are based on a publicly available real-world datasets. 
	The outcome of this survey will be the largest structured collection of various datasets and the actual datasets themselves in the form of supplementary material.
	
	The list of datasets will, for instance, cover fake profiles on social networking sites (Facebook), communication systems (Twitter), search engine link farms, auction sites, review sites, sock puppets on news sites, and various other Internet-deployed systems. 
	A key challenge is the diversity and formatting of these datasets. 
	The goal is to design a unifying format to enable scientists to easily use all available datasets for their latest research findings with minimal effort.
	
	The survey will provide a structured listing with key aspects of each dataset, such as, description, origin, size, creation date, and copyright license.

%
% The following two commands are all you need in the
% initial runs of your .tex file to
% produce the bibliography for the citations in your paper.
\bibliographystyle{abbrv}
\bibliography{sigproc}  % sigproc.bib is the name of the Bibliography in this case
\balancecolumns
\end{document}
