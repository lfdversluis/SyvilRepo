\section{Introduction}

	The sbyil attack \cite{douceur2002sybil} is nowadays a well-known attack on both centralized and decentralized systems and an active research area.
	Sybil attacks occur in a variety of systems. 
	Attacks can occur in sensor networks , social networks, content rating systems \cite{kakhki11mitigating}, vehicular ad hoc networks and many more.
	At this point in time, there are already surveys on surveys \cite{koll2014state, mohaisen2013sybil, viswanath2011analysis}.
	
	The focus of this survey will not be yet another survey on the current state of the art, but  will focus on real-world attacks using Sybil, eclipse and sinkholing techniques. 
	We perceive these to be belonging to the same broad class of attacks. 
	%The goal is to provide a list of scientific articles which are based on a publicly available real-world datasets.
	The goal is to provide a list of scientific articles and describe the datasets used in their evaluations.
	The outcome of this survey will be the largest structured collection of various datasets which are collected if the data is publicly available or if the authors are willing to share their data. Additional datasets are added as well which were either created by means of manual annotation or by other parties.\\
	The list of datasets will, for instance, cover fake profiles on social networking sites (Facebook), communication systems (Twitter), search engine link farms, auction sites, review sites, sock puppets on news sites, and various other Internet-deployed systems. 
	A key challenge is the diversity and formatting of these datasets. 
	The goal is to design a unifying format to enable scientists to easily use all available datasets for their latest research findings with minimal effort.
	
	The survey will provide a structured listing with key aspects of each dataset, including, description, origin, size, creation date, and copyright license.